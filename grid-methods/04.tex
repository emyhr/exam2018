\documentclass[__main__.tex]{subfiles}

\begin{document}

\section{Теорема об интерполяционном полиноме Ньютона}

Если $A = \left<\tau_0, \tau_1, \tau_2, ..., \tau_k\right>$ — квази-сетка $[a; b]$, то
$\Lambda_A(\tau) = (\tau-\tau_0)(\tau-\tau_1)...(\tau-\tau_k)$ - A-сеточный полином и $\mathcal(L)_k(\tau)$ — $A$ - интерполяционный полином Лагранжа для $A$-сеточной функции $\hat{A}(f)$, где $f \in \underline{C}^{k+1}([a; b], \mathbb{R}), \; A \in [a; b]$ и $R_k(\tau) = \frac{f^{(k+1)}(\xi(\tau))}{(k+1)!}\Lambda_A(\tau)$ - остаток $A$-интерпполяции Лагранжа для $A$-сеточной функции $\hat{A}(f)$. Кроме  того, полином
\begin{equation}
\mathcal{N}_k(\tau) = f(\tau_0) + f(\tau_0, \tau_1)(\tau_1 - \tau_0)(\tau_1 - \tau_0) + f(\tau_0, \tau_1, \tau_2)(\tau - \tau_0)(\tau - \tau_1) + ... + f(\tau_0, \tau_1, ..., \tau_k)(\tau - \tau_0)(\tau - \tau_1)...(\tau - \tau_{k-1})
\end{equation}
называют $A$-интерполяционным полиномом Ньютона для $A$-сеточной функции $\hat{A}(f)$
\begin{theorem}
	В условиях предыдущего:
	\begin{equation}
		f(\tau) = \mathcal{\tau} + f(\tau, \tau_0, ..., \tau_k)\Lambda_A(\tau), \text{где}\\
		\mathcal{N}_k = \mathcal{L}_k(\tau) \text{и} f(\tau, \tau_0, ..., \tau_k) = R_k(\tau)
	\end{equation}
\end{theorem}
\end{document}

\documentclass[__main__.tex]{subfiles}

\begin{document}

\section{Устойчивость конечно-разностной схемы, аппроксимирующей краевую задачу (граничные условия 1-го рода) для одномерного стационарного уравнения теплопроводности с постоянным коэффициентом теплопроводности, неотрицательной теплоотдачей (без "конвекции")}

Пусть 
\begin{equation}
	K = max\{1,\frac{(b-a)^2}{8}\}
	\label{17-1}
\end{equation}
Схема $\ref{17-1}$ из вопроса 11 устойчива и для $\forall k \in \mathbb{N}$ решение $\;^>u_{(k)}$ СЛАУ: $F_{(k)}\;>u_{(k)} = \;^>\nu_{(k)}$ удовлетворяет условию:
\begin{equation}
	||\;^>u_{(k)}|| \le K ||\;^>\nu_{(k)}||
	\label{17-2}
\end{equation}
Конечно-разностная схема имеет вид:
\begin{equation}
	F_{(k)}\;^>u_{(k)} =
	\begin{pmatrix}
		u_a\\
		f_1\\
		.\\
		.\\
		.\\
		f_{k-1}\\
		u_b
	\end{pmatrix}
	\label{17-3}
\end{equation}
Для схемы $\ref{17-3}$ рассмотрим две схемы:
\begin{equation}
	F_{(k)}\;^>x_{(k)} =
	\begin{pmatrix}
		u_a\\
		0\\
		.\\
		.\\
		.\\
		0\\
		u_b
	\end{pmatrix}
	\label{17-4}
\end{equation}
\begin{equation}
	F_{(k)}\;^>y_{(k)} =
	\begin{pmatrix}
		0\\
		f_1\\
		.\\
		.\\
		.\\
		f_{(k-1)}\\
		0
	\end{pmatrix}\\
	\label{17-5}
\end{equation}
Предполагаем, что $\;^>u_{(k)}, \;^>x_{(k)}, \;^>y_{(k)} \in \;^>\mathbb{R}^{|A_k|}(A_k)$ — решения СЛАУ $\ref{17-3}, \; \ref{17-4}, \; \ref{17-5}$. Тогда $\;^>u_{(k)} = \;^>x_{(k)} \;^>y_{(k)}$\\
$||\;^>x_{(k)}|| \le max\{|u_a|, |u_b|\}$ (лемма 2.1) из вопроса 16\\
$||\;^>y_{(k)}|| \le \frac{(b-a)^2}{8} max\{|f_1|, |f_2|, ..., |f_{k-1}|\}$ (лемма 2.2) из вопроса 16.\\
Учитывая, что $K = max\{1, \frac{(b-a)^2}{8}\}$, отсюда получаем:\\
$||\;^>u_{(k)}|| \le ||\;^>x_{(k)}|| + ||\;^>y_{(k)}|| \le K ||\;^>\nu_{(k)}||$ (см. \ref{17-3})\\
Следовательно, согласно замечанию к теореме Лакса, схема $\ref{17-3}$ устойчива.
\end{document}

\documentclass[__main__.tex]{subfiles}

\begin{document}

\section{Доказательство условной корректности явной разностной схемы численного решения задачи Коши для одномерного параболического уравнения}

Рассмотрим однородное параболическое уравнение вида: $\frac{\partial u}{\partial t}=\sigma \frac{\partial ^2 u}{\partial ^2 t}$. Рассмотрим разностную схему для данного уравнения: $$\frac{u^{n+1}_{j}-u^{n}_{j}}{\tau}=\sigma \frac{u^{n}_{j+1}-2u^{n}_{j}+u^{n}_{j-1}}{h^{2}}.$$, где $\tau$ и h- шаги соотвественно по t и x Если бы уравнение было неоднородным, мы бы просто отбросили правую часть f(x,t) , т.к. она не содержит u и не влияет на устойчивость схемы, аппроксимирующей это уравнение. Представим решение разностной схемы в виде гармоники : $u^{n}_{j}=\lambda ^n e^{i\alpha j}.$ 

Подставляя данное выражение в разностную схему получаем: $$\frac{\lambda ^{n+1} e^{i\alpha j}-\lambda ^n e^{i\alpha j}}{\tau}=\sigma \frac{\lambda ^{n+1} e^{i\alpha (j+1)}-2\lambda^n e^{i\alpha j}+ \lambda ^n e^{i \alpha (j-1)}}{h^2}$$. 

Упростим уравнение:
$\frac{\lambda-1}{\tau}=\sigma \frac{e^{i\alpha}-2+e^{-i\alpha}}{h^2}$

Знаем, что $e^{i\alpha}=\cos(\alpha)+i\sin(\alpha) $.\\
Тогда подставляя в предыдущее соотношение получим:
$$\frac{\lambda-1}{\tau}=\sigma \frac{2\cos(\alpha)-2}{h^2}$$
Или $$\frac{\lambda-1}{\tau}=\sigma \frac{-4\sin^2(\frac{\alpha}{2})}{h^2}$$

Тогда $\lambda=1-\frac{4\sigma\tau}{h^2}\sin^2(\frac{\alpha}{2})$. Условие устойчивости разностных схем: $|\lambda|\leq 1$. Тогда $-1\leq 1-\frac{4\sigma\tau}{h^2}\sin^2(\frac{\alpha}{2}) \leq 1$. Если $\sigma\geq 0$, то правое условие выполняется автоматически. Поэтому рассмотрим $\frac{\tau}{h^2}\sin^2(\frac{\alpha}{2})\leq \frac{1}{2\sigma}$. Но число $\alpha $ - комплексное, нам необходимо его убрать. Для этого перейдем к более строгому условию : $$\frac{\tau}{h^2}\leq\frac{1}{2\sigma}$$
И в итоге получаем соотношение : $ \frac{2\tau\sigma}{h^2}\leq 1 $ \\
Таким образом, при выполнении этого условия, разностная схема будет корректна, так как выполняется устойчивость и схема аппроксимирует уравнение.
\end{document}
